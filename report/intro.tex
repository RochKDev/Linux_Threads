\section{Introduction}

Lors de la conceptualisation d'un programme, il peut arriver qu'à certains moment de celui-ci nous ayons besoin de déléguer certaines actions à un autre processus afin de ne pas bloquer l'exécution entière de notre programme. Les deux solutions qui s'offre alors à nous sont; le \textit{fork} d'un processus ainsi que les threads. A premier abord ces deux solutions semble fortement se ressembler, mais quand on regarde de plus près il y as plein de petite différence qui peuvent complètement changer le fonctionnement de notre programme. 
C'est pour cela que dans ce travail de recherche, nous allons nous pencher sur les différences et ressemblances majeures entre les threads et les processus. Nous rentrerons un peu plus en détails pour les threads car ils sont souvent moins connus ainsi qu'un peu plus complexe que les processus. L'objectif final de cette recherche est de mieux comprendre les différences/similitudes de ces 2 systèmes, mieux comprendre leurs fonctionnement, de voir si un système est supérieur à l'autre ainsi que de bien choisir le système le plus adéquat à la situation. 