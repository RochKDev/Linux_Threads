\section{Création de Thread}
Dès lors que nous savons ce qu'est un thread, nous pouvons maintenant nous intéresser à son implémentation concrète dans nos programmes. Pour ce faire, nous devons tout d'abord créer ledit thread. La création d'un thread s'effectue via l'appel de fonction \textit{pthread\_create()}.
\\

\subsection{La fonction \textit{pthread\_create()}}
\begin{lstlisting}[ title = Création d'un thread]
#include <pthread.h>

int pthread_create(pthread_t *thread, const pthread_attr_t *attr,
 void *(*start)(void *), void *arg);
\end{lstlisting}
\vspace{\baselineskip}


La fonction renvoie un entier pour signaler si la création s'est bien passée. Ici, nous rencontrons une spécificité des pthreads, à savoir que la fonction \textit{pthread\_create()} retournera 0 en cas de réussite, mais elle retournera un entier positif en cas d'échec. Contrairement aux fonctions standard qui renvoient une valeur négative en cas d'échec.
\\

Continuons en listant les paramètres que nous devons passer à la fonction. Le premier paramètre est un pointeur vers un buffer de type \textit{pthread\_t} dans lequel sera stocké l'ID unique du thread créé. Le deuxième paramètre est un pointeur vers un objet \textit{pthread\_attr\_t} qui spécifie les différents attributs du thread créé. Si on passe \textit{NULL} à \textit{attr}, alors le thread possédera des attributs aléatoires.

Le troisième paramètre, le plus difficile à comprendre, est « \textit{void *(*start) (void *)} ». Analysons un peu cela. Il s'agit ici d'une déclaration de pointeur de fonction. Décortiquons chaque élément afin de mieux comprendre :



\begin{itemize}
    \item 	void *: Le type de retour de la fonction. Ici, la fonction renvoie un pointeur vers void, ce qui signifie qu’elle peut pointer vers n’importe quel type de données.
    \\
    \item 	(*start) : start est un pointeur vers une fonction. Les parenthèses sont nécessaires pour indiquer que start est un pointeur vers une fonction, et non une fonction qui renvoie un pointeur.
    \\
    \item 	(void *) : Le type d’argument que la fonction accepte. Ici, la fonction accepte un pointeur vers void, ce qui signifie qu’elle peut accepter un pointeur vers n’importe quel type de données.
    \\   
\end{itemize}
\vspace{\baselineskip}

Pour finir, le dernier paramètre est un pointeur vers void. Il s'agit des arguments à passer à la fonction \textit{start}.Vu que nous avons affaire à un pointeur de \textit{void}, nous pouvons passer n'importe quel objet à \textit{start}; nous pouvons également passer \textit{NULL}. Généralement, \textit{arg} pointe vers une variable globale ou une variable sur le tas. Si nous avons besoin de passer plusieurs arguments à \textit{start}, nous pouvons passer à \textit{arg} un pointeur vers une structure contenant les différents arguments en tant qu'emplacement séparé.
\\

\subsubsection{Exemple de code de création}
\begin{lstlisting}[title = Exemple de code de création ]
#include <stdlib.h>
#include <stdio.h>
#include <string.h>
#include <stddef.h>
#include <pthread.h>
void* myturn(){
	for(int i = 0; i < 8; i++){
		sleep(1);
		printf("my Turn\n");
	}
}
void* yourturn(){
	for(int i = 0; i < 4; i++){
		sleep(1);
		printf("Your Turn\n");
	}
}

int main(){
	pthread_t newthread;

	pthread_create(&newthread, NULL, &myturn, NULL);
	
	yourturn();
}
\end{lstlisting}
\vspace{\baselineskip}


    Comme on peut le remarquer, on passe, dans l'ordre, la référence au buffer pour récupérer l'id du thread,
ensuite on a marqué \textit{NULL} pour les attributs du thread car cela ne nous intéresse pas, le prochain argument est le pointeur vers notre fonction \textbf{Myturn}. Pour finir, le dernier argument est la liste des arguments que l'on souhaite passer à notre fonction \textbf{Myturn}, dans notre cas \textit{NULL} car nous n'avons pas besoin de paramètres pour cette fonction.

\subsubsection{La fonction \textit{gettid()}}

Lors de la création d'un thread, comme nous venons de le voir, son identifiant seras stocké dans le buffer destiné a cet usage dans le thread appelant. Mais il se pourrais que le thread aie besoin de son propre ID. Dans ce cas, la fonction \textit{gettid()} entre en jeu. Tout comme son équivalent du coté des processus, \textit{getpid()}, elle sert à connaitre l'id du thread appelant.
\begin{lstlisting}
#define _GNU_SOURCE
#include <unistd.h>

    pid_t gettid(void);

\end{lstlisting}
